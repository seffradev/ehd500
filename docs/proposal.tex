\documentclass[12pt]{article}
\usepackage[
    a4paper,
    top=1cm,
    bottom=2cm,
    left=3cm,
    right=3cm,
    headheight=12pt,
    includehead,
    includefoot,
    heightrounded
]{geometry}
\usepackage{graphicx}
\usepackage[natbib=true, style=numeric, sorting=none]{biblatex}
\usepackage{fancyhdr}
\usepackage[dvipsnames]{xcolor}
\usepackage{parskip}
\usepackage{titlesec}
\usepackage{csquotes}
\usepackage{lastpage}
\usepackage{amsmath}
\usepackage[english]{babel}
\usepackage[style=iso, en-GB]{datetime2}
\usepackage{svg}
\usepackage{titling}
\usepackage{adjustbox}

\addbibresource{references.bib}

\renewcommand{\headrulewidth}{0.4pt}
\renewcommand{\footrulewidth}{0.4pt}

\titleformat{\section}
  {\normalfont\fontsize{12}{12}\bfseries}{\thesection}{1em}{}

\titleformat{\subsection}
  {\normalfont\fontsize{12}{12}\itshape}{\thesubsection}{1em}{}

\pagestyle{fancy}
\fancyhf[L,R,O,E]{}
\fancyhead[C]{\textbf{Proposal -- Degree project}}
\fancyhead[R]{\thepage(\pageref{LastPage})}
\fancyfoot[L]{
Högskolan Väst, Institutionen för Ingenjörsvetenskap,
\newline
Avdelningen för Datateknik, 461 86 Trollhättan
}
\fancyfoot[R]{
\adjincludegraphics[width=0.25\textwidth,valign=M]{images/university-west-logo.jpg}
}

\begin{document}

\thispagestyle{fancy}

\section*{Title of project}
An alternative implementation to SIM over-the-air -- A utility for distributed mobile network testing

\section*{Author}
Hampus Avekvist,\newline
\verb|hampus.avekvist@hey.com|,\newline
+46 76 035 22 26,\newline
Computer Engineering -- programming and network technology, 180 hp

\section*{Company}
Leissner Data AB,\newline
Nohabgatan 11H, SE-461 53 Trollhättan

\section*{Supervisor at company}
Mikael Johansson,\newline
\verb|mj@leissner.se|,\newline
+46 70 180 95 20

\section*{Project background}
Scaled up version of \cite{osmo-remsim}.

\begin{itemize}
    \item TODO: Define SIM
    \item TODO: Define APDU
\end{itemize}

\section*{Problem formulation}

The thesis entails the creation and documentation of the following
components:
\begin{itemize}
    \item SIM-to-IP device
    \item IP-to-radio interface device
    \item SIM-radio interface connectivity hub
    \item Graphical user interface
\end{itemize}
Elaboration of each component is available below.

\subsection*{SIM-to-IP device}
This component enables the transmission of APDU:s to- and from
the SIM-side over an IP network. The device uses a smart card
reader and a programmable IP-enabled general purpose computer
(e.g. a Raspberry Pi).

\subsection*{IP-to-radio interface device}
This component enables the transmission of APDU:s to- and from
the mobile side over an IP network.

\subsection*{SIM-radio interface connectivity hub}
This component orchestrates the connections between different SIM
and radio interfaces, allowing switching of one SIM to another.

\subsection*{Graphical user interface}
This component provides an administration interface for the
connectivity hub. An initial design is a list of dropdowns, where
each dropdown refer to a radio interface and the possible
selections available in the dropdown are free SIM cards to connect
to.

The creation of a utility device that enables a SIM, connected
through a smart card reader, to communicate with a mobile network
enabled device over IP.

The utility, aside from the physical components connecting a radio
interface to a SIM card, will include a service for switching the
SIM card on demand via a graphical user interface.

TODO: Describe the task or question you are going to solve.

TODO: State realistic and measurable goals.

\section*{Limitations (if applicable)}
If you want to emphasize something that should not be included
in your work, state these things here.

\section*{Methodology}
TODO: Describe how you will carry out your work. Who will do what?

\section*{Resource plan (if applicable)}
Equipment required will be provided by Leissner Data AB.
The following will (minimally) be used in the project:

\begin{itemize}
    \item A couple of smart card readers
    \item A couple of user equipment (modems, mobile phones)
    \item Two Raspberry Pi
\end{itemize}

\printbibliography[
    heading=bibintoc,
    title={References}
]

\end{document}
