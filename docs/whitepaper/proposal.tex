\documentclass[12pt]{article}
\usepackage[
    a4paper,
    top=1cm,
    bottom=2cm,
    left=3cm,
    right=3cm,
    headheight=12pt,
    includehead,
    includefoot,
    heightrounded
]{geometry}
\usepackage{graphicx}
\usepackage[natbib=true, style=numeric, sorting=none]{biblatex}
\usepackage{fancyhdr}
\usepackage[dvipsnames]{xcolor}
\usepackage{parskip}
\usepackage{titlesec}
\usepackage{csquotes}
\usepackage{lastpage}
\usepackage{amsmath}
\usepackage[english]{babel}
\usepackage[style=iso, en-GB]{datetime2}
\usepackage{svg}
\usepackage{titling}
\usepackage{adjustbox}

\addbibresource{references.bib}

\renewcommand{\headrulewidth}{0.4pt}
\renewcommand{\footrulewidth}{0.4pt}

\titleformat{\section}
  {\normalfont\fontsize{12}{12}\bfseries}{\thesection}{1em}{}

\titleformat{\subsection}
  {\normalfont\fontsize{12}{12}\itshape}{\thesubsection}{1em}{}

\pagestyle{fancy}
\fancyhf[L,R,O,E]{}
\fancyhead[C]{\textbf{Proposal -- Degree project}}
\fancyhead[R]{\thepage(\pageref{LastPage})}
\fancyfoot[L]{
Högskolan Väst, Institutionen för Ingenjörsvetenskap,
\newline
Avdelningen för Datateknik, 461 86 Trollhättan
}
\fancyfoot[R]{
\adjincludegraphics[width=0.25\textwidth,valign=M]{images/university-west-logo.jpg}
}

\begin{document}

\thispagestyle{fancy}

\section*{Title of project}
An implementation of a remote SIM bank -- A utility for
configuration, reprogramming and distributed coverage testing

\section*{Author}
Hampus Avekvist,\newline
\verb|hampus.avekvist@hey.com|,\newline
+46 76 035 22 26,\newline
Computer Engineering -- programming and network technology, 180 hp

\section*{Company}
Leissner Data AB,\newline
Nohabgatan 11H, SE-461 53 Trollhättan

\section*{Supervisor at company}
Mikael Johansson,\newline
\verb|mj@leissner.se|,\newline
+46 70 180 95 20

\section*{Project background}

Subscriber Identity Module (SIM) is a smart card for telecommunication.
Like other smart cards they communicate with a reader, like a mobile
phone or a modem, by sending application protocol data units (APDU:s).
This project intends to build a utility that can intercept, manipulate
and forward APDU:s. The forwarding is intended over an IP network to
allow having user equipment (e.g. a modem or a mobile phone) in one
location with the SIM card in another.

An example for when this is useful is in roaming testing services
from a mobile network operator (MNO), as well as mobile virtual
network operator (MVNO). Existing work which inspires this has been
made by Osmocom \cite{osmocom-remsim}. In \cite{alghawi-simbox}, the technology
is portrayed for another possible purpose (albeit illegal) but
illustrates system capabilities and example equipment.

\section*{Problem formulation}

The thesis entails the creation and documentation of the following
components:
\begin{itemize}
    \item SIM-to-IP device
    \item IP-to-radio interface device
    \item SIM-radio interface connectivity hub
    \item Graphical user interface
\end{itemize}
Elaboration of each component is available below.

\subsection*{SIM-to-IP device}
This component enables the transmission of APDU:s to- and from
the SIM-side over an IP network. The device uses a smart card
reader and a programmable IP-enabled general purpose computer
(e.g. a Raspberry Pi).

\subsection*{IP-to-radio interface device}
This component enables the transmission of APDU:s to- and from
the user equipment side over an IP network. Powered by an
IP-enabled general purpose computer (e.g. a Raspberry Pi)
connected to the user equipment.

\subsection*{SIM-radio interface connectivity hub}
This component orchestrates the connections between different SIM
and user equipment, allowing switching of one SIM to another. It
also provides manipulation, interception and capture of APDU:s.

\subsection*{Graphical user interface (GUI)}
This component provides an administration interface for the
connectivity hub. An initial design is a list of dropdowns, where
each dropdown refer to user equipment and the possible selections
available in the dropdown are free SIM cards to connect to.

\section*{Limitations}
The project will focus on enabling roaming testing and will, if
time allows, extend to manipulation, interception, packet capture
and reprogramming. The aforementioned four components are viable
even in this case and primarily the connectivity hub and GUI would
be expanded.

\section*{Methodology}

The preparatory work will contain planning and experimentation of
different implementations for the library enabling smart card
connectivity \cite{pcsc}. This experimentation is due to
observed limitations in some implementation of the library from
previous work, and this will drive the language choice for the
SIM-interfacing parts.

An initial implementation with a singular user equipment and
singular SIM will be constructed, where communication flows over
IP. This doesn't require a proper connectivity hub nor a GUI.
After that, implementation of a connectivity hub will be started
and this enables the scaling to multiple SIM and devices.

Design of hardware and software inspired by \cite{osmocom-remsim}
will be conducted. Studying of technical specifications regarding
APDU:s and SIM cards is a necessity to enable reprogramming and
functional APDU manipulation. These will primarily be from ISO,
3GPP and ETSI.

\section*{Resource plan}
Equipment required will be provided by Leissner Data AB.
The following will (minimally) be used in the project:

\begin{itemize}
    \item A couple of smart card readers
    \item A couple of user equipment (modems, mobile phones)
    \item Two Raspberry Pi
\end{itemize}

\printbibliography[
    heading=bibintoc,
    title={References}
]

\end{document}
