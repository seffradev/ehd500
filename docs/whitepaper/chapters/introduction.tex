\chapter{Introduction}

Connectivity is essential for modern society. The ability to keep in
touch, share knowledge and collaborate remotely is growing evermore
important. A fundamental group of technologies and infrastructure
enables these facets, all categorized within the telecommunications
umbrella. Telecommunication businesses around the world compete to
provide the best service, while assisting one another to enable a
greater reach.

For a user of these technologies and infrastructure, identification
is required to ascertain the authenticity and capabilities that
should be provided. This is provided through a Subscriber Identity
Module (SIM) \cite{etsi-ts-131-102}, a type of smart card \cite{etsi-ts-102-221},
that stores encryption keys and application files utilized to verify
the subscriber in the telecommunications network. These SIM may have
varying support when roaming or if a device is 4G- or 5G-enabled.

Considering the varying support for SIM configurations in different
telecommunication networks, there is an interest for utilities that
can easily enable testing, configuration and manipulation of SIM:s,
preferably with user equipment (UE) in arbitrary locations and a bank
of SIM in an easily (though not unauthorized) accessible office space
that allows hot-swapping of the SIM connected to the differing UE.

The goal of this work is to develop said utilities that provide a
SIM bank, user equipment connectivity and a mediation service. The
mediation service enables pairing of SIM-to-UE with a logical
hot-swapping facility, rendering physical access to the SIM bank
only occasionally required rather than consistently.

The report starts off by introducing previous work in this sector,
formally introducing the problem and then provides necessary theory
and background information. The second chapter delves into the
methodology used in the design and implementation, alongside how
decisions were made and what material that was utilized. It also
shows the information collection process for relevant technical
specifications studied, used for the theory and background sections
in this chapter and the same information that guides the
implementation.

The third chapter addresses the results with a complete description
of the implementation alongside diagrams, measurement results from
the benchmarks and functionality tests described in \ref{sec:experimentation}.
In the fourth chapter, an analysis based on the implementation is
performed, addressing what could've been done differently and how
the tools, structure and method aided in the solution for the
problem at hand. Ethical considerations are provided and in the
fifth chapter conclusions are drawn.

\section{Background}

Leissner Data AB is a Swedish telecommunications firm from
Trollhättan founded in 1969 \cite{leissner-about}. The company
was not originally in telecommunications but pivoted to an 
Internet Service Provider (ISP) role in 1996 with the founding of
the sister company Götalandsnätet AB. In 2001 Leissner started
offering telecommunication systems and has since extended to mobile
platforms. These mobile platforms are, for example, complete
virtual mobile core infrastructure such as Packet Delivery Network
Gateways (PGW:s), Short Message Service Centers (SMSC:s) and Home
Subscriber Servers (HSS:s) \cite{leissner-mobile-core}.

After a partnership with Hi3G \cite{leissner-about}, Leissner
became a Mobile Virtual Network Enabler (MVNE) and in their
product catalogue, they sell SIM to the Mobile Virtual Network
Operators (MVNO). The aforementioned SIM:s is equipped with a
digital profile compatible with their systems. These digital
profiles are preset file structures and contents containing
information such as the International Mobile Subscriber Identifier
(IMSI), Service Provider Name (SPN) and encryption keys.

The digital profile may contain files that can be changed \cite{placeholder-source-for-smart-card-profiles},
e.g. the SPN or the Personal Identification Number (PIN) as well as
files that may not be changed, e.g. IMSI and Integrated Circuit
Card Identifier (ICCID). The design of the digital profiles is up
to the agreement between SIM supplier and the operator where which
files should be writeable or not can be decided.

Considering the aforementioned, a desire for a simplified test
platform arose. This test platform would allow hot-swapping of
SIM in a nearby location while having user equipment connecting
from a remote location, e.g. in another city or at a customer
office to test reliability and function of the mobile core
platform with different SIM profiles.

Functionality requested in such a platform include tracing of
application protocol data units (APDU:s) where the transmitted
APDU:s may be sniffed, or manipulation where transmitted APDU:s
are altered to e.g. transform one PIN into another or by changing
an SPN to something else. Reprogramming of SIM is another
requested capability which could replace existing software and
offer all SIM-related functionality inside a single system.

Lastly, support for other kinds of SIM than just physical in a
smart card reader is requested. This extends to virtual SIM in
a custom-developed runtime.

\section{Problem formulation}
\label{sec:problem-formulation}

The intention of this thesis is to implement a couple of technical
solutions addressing the requests from Leissner Data. The requests
can be dissolved into the following sections.

\subsection{Functionality}

Primary requested functionality is the ability to forward APDU:s
between a SIM and UE. This renders forwarding the initial problem
to solve and is dependent upon by secondary functionalities.
Tracing APDU:s and the ability to manipulate APDU:s extend the
forwarding functionality by either logging or changing the data.

Reprogrammability entails generating APDU:s from an alternative
source than from user equipment being sent directly to the SIM and
extends requirements to available functionality in a user
interface. This is out of scope for this project.

\subsection{Scalability}

Scalability in this case intends on providing an ability to add and
remove both SIM banks and UE from the system without configuration
changes in a central service. Both the SIM banks and the UE may be
in arbitrary locations.

\subsection{Security considerations}

The APDU:s may contain confidential information such as encryption
keys and identification numbers, normally not exposed due to the
locality of the SIM inside the UE but in the remote forwarding case
requiring hiding. Connectivity of the SIM bank and the UE to the
full system must also be handled in a secure way to not allow any
device to connect and get APDU delivered.

\subsection{Commercialization}

The nature of being able to test mobile core products on different
SIM profiles is not limited to only the developers of the products
themselves but also the business customers buying and operating the
products. Therefore, there is an interest in having a sellable
service and easily deployed devices a customer can order.

\subsection{Delimitations}

Manipulation of APDU:s, due to the added complexity in interface
design and time constraints will not be considered in this draft.
Similarly, the ability to reprogram SIM are out of scope for this
work as that relates to functionality not directly coupled with
the forwarding of APDU:s to and from a SIM and UE.

Only SIM will be considered in this implementation, excluding
general smart cards since the work is aimed at telecommunications.

\section{Previous work}

The primary inspiration and function description for this project
is Osmocom's remSIM project \cite{osmocom-remsim}. It is a
complete open source implementation of a SIM bank with forwarding
capabilities to user equipment. Osmocom provides a user manual,
open hardware and source code that is studied for this project's
implementation.

There are additional solutions in the field of SIM banks with
similar capabilities to the Osmocom remSIM and the proposed
solution described in this document. These solutions, however, are
proprietary \cite{polygator-sim-bank} complete solutions, usually
offering a cloud service they own \cite{placeholder-source-cloud-sim-bank}
not enabling the capabilities required for Leissner Data's purposes.
They may also restrict the communications protocol to the SIM bank
to a proprietary, undocumented protocol rendering implementation
difficult.

The existing solutions miss out on some of the requested
functionality and while they could be extended upon, Leissner Data
aims to develop their own solution to have full control of their
offering.

\section{Theory}

This section intends to elaborate on the necessary theory to
understand the utilization of SIM in the context of the
implementation and the related nomenclature.

\subsection{Nomenclature}

Definitions and descriptions of terminology commonly used in this
report are provided here. A word is explicitly stated if it has
a specific definition in regards to this document.

\subsubsection{User Equipment}

User Equipment (UE) is what in regular telecommunication contexts
may be used by a human, e.g. a mobile phone or wireless modems.
They may also be called mobile equipment or mobile stations.

\subsubsection{Subscriber Identity Module}

A SIM is a type of smart card utilized for the telecommunications sector
\cite{etsi-ts-131-102}. Historically it was called SIM during the prime of 2G,
but a later revision called Universal SIM (USIM) with 3G \cite{etsi-ts-131-102}
support was developed. USIM also enable 4G and 5G support given an
appropriate digital profile. While USIM may be the correct term, SIM
will be used in its place, though colloquially mean USIM.

A SIM is used to provide authentication capabilities by storing
encrypted keys and identification files within a telecommunications
application \cite{etsi-ts-131-102}. The SIM authenticates a
specific subscriber and their capabilities in a mobile network.

\subsubsection{Application Protocol Data Unit}

Application Protocol Data Units (APDU:s) are the application-level
communications protocol, residing on top of the transport layer of
a smart card and terminal interface \cite{etsi-ts-102-221}. These
may be file selections in the smart card, file reads, writes or
some other operation supported by the smart card \cite{etsi-ts-102-221}.

\subsubsection{SIM bank, SIM box or SIM farm}

A SIM bank is a device that provides connectivity to multiple SIM
simultaneously \cite{hyprms-sim-bank}. This may be used for
connecting to multiple operators or to have different SIM
configurations from a singular operator. One or more devices may
connect to the bank and choose which SIM to use, from which the
devices can act as user equipment.

The terms SIM bank, box and farm all refer to the same thing and
may be used interchangeably. Different sources use the different
phrases but SIM bank will used in this paper.

\subsubsection{SIM consumer}

SIM consumer in relation to this work is the device that receives
forwarded APDU:s from a remote SIM bank. It is also this device
that is connected to user equipment and therefore forwards the
received APDU:s to the UE.

\subsubsection{Mediator}

The mediator is in this paper the service that manages connections
between SIM:s in the SIM bank and ties them to a SIM consumer. It
manages Answer To Resets (ATR:s) for each SIM so user equipment
can identify the type of card upon connection establishment.
Alongside ATR:s, it forwards APDU:s to support tracing and
manipulation capabilities.

\subsection{Requirements engineering}

This section cites from Sommerville's work Software Engineering \cite{sommerville-software-engineering}.
Requirements engineering is the craft of finding, defining,
validating and changing requirements for the solution of a problem.
Requirements grow and change alongside the lifecycle of a project
and requirements engineering is a formalization of how the process
works. Requirements can be separated into functional and
non-functional where functional requirements specify what a system
should do while non-functional requirements specify peripheral
qualities of the system, such as implementation constraints,
response time and interface design.

The first step is elicitation, which includes gathering of
requirements from stakeholders, establishing the kind of
requirement, prioritizing and informally documenting said
requirement. Stakeholders are people with legitimate interest in
the system in question, accounting for members of the development
team, customers, operational staff and others that are directly
involved or affected by the system. 

The second step is specification. This entails writing formal or
more technical specifications with additional details relevant to
the implementers but not all stakeholders. Formalizations may take
different appearances, as shown in figure 4.11 in \cite{sommerville-software-engineering}
where natural language, both in unstructured and structured form,
graphical notations and mathematical models are examples of
formalizations.

The third step is validation of the requirements. This means
testing the requirements for validity, consistency, completeness,
realism and verifiability. The step is vital to ascertain the
requirements are correct in the scope of the problem at hand and
that a proposed solution fulfilling the requirements also solves
the problem.

The last step is the change of requirements, specifically
management of changes. Letting requirements change is important
because new insights, knowledge and changes in the surroundings
spring up which affect the problem space. Key points in managing
changes is to provide traceability, which entails tracking of
relevant stakeholders for a requirement, how the requirement has
changed and why. The relationships between requirements need also
be tracked to identify peripheral changes when a requirement is
updated.
