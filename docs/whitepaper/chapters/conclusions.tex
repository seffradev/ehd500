\chapter{Conclusions}

\section{Future work}

\subsection{Experimentation}
\label{sec:experimentation}

In tandem with prototyping, experiments testing multiple
implementations of the PC/SC \cite{pcsc} library in different
programming languages have been conducted. The experiments are to
guide technical decisions for what language to use depending on
which implementation that fulfills requirements for communications
with smart cards. Four languages were chosen, JavaScript (on
NodeJS) and Python, to compare two higher-level languages, as well
as C++ and Rust to compare two lower-level languages.

The experiments test the performance and ability to communicate
with multiple smart cards at the same time, when transmitting
multiple APDU:s in quick succession.

Idea: Elaborate on what will be tested and what measurements will
be taken and why.

Note: This section is a stub and needs additional information.

\subsection{Implementation}
\label{sec:implementation}

As already mentioned in chapter
\ref{sec:software-requirements-engineering}, tests will be used to
ensure requirements are being upheld. This leads swiftly into a
test-driven development approach
\cite[242-245]{sommerville-software-engineering} while system
design will be based around the non-functional requirements. The
process is iterative as mentioned in the previous sections and
shown in Fig.~\ref{fig:process-diagram}, where new findings in the
prototyping or experimentation stages drive changes in the main
implementation. The implementation step will equally drive a need
to return to the prototype, experimentation and a change of
requirements.

\begin{figure}[ht]
	\centering
	\includegraphics[width=0.5\textwidth]{images/process.mmd.png}
	\caption{The iterative development process. The dashed lines
	implies an optional order of execution}
	\label{fig:process-diagram}
\end{figure}

Throughout development, designs decisions are documented in
architectural decision records (ADR:s) \cite{adr, adr-github} that
contain related requirements. Said ADR:s will be structured,
describing what ideas were considered, which one was chosen and why
to allow historical transparency in the system design.
