\chapter{Methodology}

This chapter elaborates on the material, equipment and methods used
to derive a solution to the problem statements in chapter~\ref{sec:problem-formulation}.

\section{Materials and equipment}
\label{sec:materials-and-equipment}

This section lists the equipment, material and tools used in the
implementation. Fig.~\ref{fig:simplified-system-diagram} shows how
the majority of the physical components are connected.

\begin{figure}[ht]
	\centering
	\includegraphics[width=\textwidth]{images/system-simplified.mmd.png}
	\caption{Overview of the intended system diagram}
	\label{fig:simplified-system-diagram}
\end{figure}

\subsection{SIMtrace 2}

The SIMtrace 2 by Osmocom is an open hardware platform with open
source software \cite{simtrace}. It comes preloaded with SIM
tracing firmware called \verb|trace| \cite{simtrace-wiki} but
additional firmware for smart card emulation exists, provided by
Osmocom. This firmware is called \verb|cardem|
\cite{simtrace-wiki} and is used for communicating with user
equipment, sending APDU:s either constructed by software or
forwarded from a remotely accessed smart card. The hardware is
capable of tracing and emulating all kinds of ISO 7816
\cite{etsi-ts-102-221} smart cards when the T=0 protocol is used
\cite{simtrace-wiki} but the focus is on SIM.

\subsection{Raspberry Pi 4}

Two Raspberry Pi 4 units are used due to the ubiquity and
portability of the hardware platform. The main computer
requirements based on the proposed system are two Linux-based
devices with USB and IP network connectivity capabilities,
rendering most computer systems a viable choice where the choice
of Raspberry Pi is due to the small form factor, inexpensive
hardware and lower power drain, though none of those properties
are a requirement.

\subsection{Teltonika RUT200 v2.0}

The Teltonika RUT200 v2.0 is used as provisional user equipment
the SIMtrace 2 hardware running \verb|cardem| connects to, meaning
the receiver of forwarded APDU:s from a remotely accessed smart
card. The device provides 2G, 3G and 4G capabilities for mobile
network connectivity, authenticated to via the remotely accessed
smart card and has a WiFi and Ethernet interface allowing wired
and wireless connections and internet access, acting as a gateway
between the mobile network and the local network.

\subsection{Smart card readers, SIM, server, and router}

Generic physical smart card readers are used for enabling access to
smart cards for the Raspberry Pi which will forward communications
from the smart card over IP to a destination device, e.g. a server
or a peer Raspberry Pi connected to a SIMtrace unit and user
equipment.

A server, virtual or not, is used as a central connectivity hub
enabling rerouting of SIM-to-UE communication to a either a
different UE or a different SIM. A basic router will be used to
provide IP network access for the Raspberry Pi and internet access
to the aforementioned connectivity hub.

\subsection{Automated tooling}

Tools for documentation generation will be utilized, both for
automatic generation of changelogs that document available
functionality in an operational perspective but also tools that
generate software library documentation from code comments.
Examples of such tools are Doxygen \cite{doxygen} for software
library documentation generation and git-cliff \cite{git-cliff} for
changelog generation.

To enable changelog generation, git-cliff requires a consistent
format of git commits \cite{git-cliff} called Conventional Commits
\cite{conventional-commits}. This, using git-cliff also enables
automatic version numbering \cite{git-cliff-bump-version} by
following the Semantic Versioning (SemVer) specification
\cite{semver}. If required, additional commit types will be created
and documented with custom git parsers added to git-cliff \cite{git-cliff-tips-and-tricks}.

Automated testing will be conducted to ensure that requirements are
upheld. The testing will be implemented using GoogleTest \cite{googletest}
and Google Mock \cite{googletest}. The tests will run on the
continuous integration/continuous deployment (CI/CD) platform
GitHub Actions \cite{github-actions} to ensure that no regressions
have been committed. Google Test will be used for authoring of
tests and Google Mock is utilized for mocks used by the tests.

Performance regressions will be detected by the use of automated
benchmarking. The primary implementation will be via Google's
microbenchmark support library \cite{google-benchmark} for smaller
units and \verb|perf| \cite{perf} will be used for full application
benchmarking if necessary. \verb|perf| will, however, not be
automated.

\section{Prototype development}
\label{sec:prototype-development}

The initial development is for explorative purposes, to create a
prototype \cite[62]{sommerville-software-engineering} \cite[56]{thomas-hunt-pragmatic-programmer}
where ideas lead to requirements, what to test for in the
benchmarking and functionality experiment listed in section~\ref{sec:experimentation}
which alongside the TARA explained in section~\ref{sec:tara} drives
the decisions behind the final implementation. Requirements
engineering and experimentation, elaborated on below, will use
knowledge acquired in the prototyping phase while also themselves
provide additional information that can sprout new concepts to
initially implement in the prototype before adding to the final
implementation.

\begin{figure}[ht]
	\centering
	\includegraphics[width=0.9\textwidth]{images/prototype.mmd.png}
	\caption{Prototype system overview}
	\label{fig:prototype-diagram}
\end{figure}

The prototype, as shown in Fig.~\ref{fig:prototype-diagram}, only
requires the functionality for communication between the two
Raspberry Pi:s used, while being restricted to only a singular SIM
and singular UE. The communication protocol between the two
computers is a simple insecure protocol on top of TCP instead of
anything that may be proposed in the TARA. The security
considerations are not as important in the lab environment the
prototype is developed within and has no customer-sensitive
information, allowing an insecure communications protocol.

\section{Experimentation}
\label{sec:experimentation}

In tandem with prototyping, experiments testing multiple
implementations of the PC/SC \cite{pcsc} library in different
programming languages have been conducted. The experiments are
to guide technical decisions for what language to use depending on
which implementation that fulfills requirements for communications
with smart cards. Four languages were chosen, JavaScript (on
NodeJS) and Python, to compare two higher-level languages, as well
as C++ and Rust to compare two lower-level languages.

The experiments test the performance and ability to communicate with
multiple smart cards at the same time, when transmitting multiple
APDU:s in quick succession.

Idea: Elaborate on what will be tested and what measurements will
be taken and why.

Note: This section is a stub and needs additional information.

\section{Threat analysis and risk assessment}
\label{sec:tara}

The threat analysis and risk assessment (TARA) \cite{tara} directly
responds to the security considerations in section~\ref{sec:problem-formulation}
by, alongside STRIDE \cite{stride}, providing an analytical
framework to identify and classify security risks. The TARA is
conducted on the intended system overview shown in Fig.~\ref{fig:simplified-system-diagram}.

\section{Software requirements engineering}
\label{sec:software-requirements-engineering}

Requirements elicitation will be used in tandem with the prototype
development and experimentation to narrow down and define what the
final product should look like. Specification of requirements will
use the collected knowledge and experience from prototyping and
experimentation both in structured natural language \cite[121]{sommerville-software-engineering},
as well as automated tests where possible \cite{test-cases-as-requirements}.
Validation will happen in accordance with the following sections
and change management will be handled by version control management
(using \verb|git| \cite{git}) and structured documents listing
stakeholders, what and why, as well as assigning a requirement
identifier to each requirement. By using \verb|git|, each commit
can contain relevant information on why a requirement was changed.

\subsection{Compilation of specifications}

Aside from the requirements for the software solution, the
specifications \cite{etsi-ts-102-221, etsi-ts-131-102} 
themselves pose practical requirements for a functional system.
Therefore, they are an additional source of non-functional
requirements. Following the specifications also provide a
validation angle for the authored requirements.

\subsection{Existing solutions and usage}

Existing solutions such as \cite{osmocom-remsim} provide knowledge
to compare with regarding how to solve the fundamental functional
requirements aiding in validation of the system requirements. The
existing material may also provide additional information not found
in the specifications, giving a fuller picture of system
requirements.

\section{Implementation}
\label{sec:implementation}

As already mentioned in chapter \ref{sec:software-requirements-engineering},
tests will be used to ensure requirements are being upheld. This
leads swiftly into a test-driven development approach \cite[242-245]{sommerville-software-engineering}
while system design will be based around the non-functional
requirements. The process is iterative as mentioned in the previous
sections and shown in Fig.~\ref{fig:process-diagram}, where new
findings in the prototyping or experimentation stages drive changes
in the main implementation. The implementation step will equally
drive a need to return to the prototype, experimentation and a change
of requirements.

\begin{figure}[ht]
    \centering
	\includegraphics[width=0.5\textwidth]{images/process.mmd.png}
	\caption{The iterative development process. The dashed lines implies an optional order of execution}
	\label{fig:process-diagram}
\end{figure}

Throughout development, designs decisions are documented in
architectural decision records (ADR:s) \cite{adr, adr-github} that
contain related requirements. Said ADR:s will be structured,
describing what ideas were considered, which one was chosen and why
to allow historical transparency in the system design.
