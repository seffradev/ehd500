\chapter{Methodology}

\section{Literature study}

\subsection{Compilation of specifications}

To identify the capabilities and technologies usable for
implementation.

\subsection{Existing solutions and usage}

To provide examples of what has been done, how it may be used and
practical notes possibly missed from the specifications.

\section{Materials and equipment}
\label{sec:materials-and-equipment}

This section lists and elaborates on the equipment and material
used in the implementation.

\subsection{SIMtrace 2}

The SIMtrace 2 by Osmocom is an open hardware platform with open
source software \cite{simtrace}. It comes preloaded with SIM
tracing firmware called \verb|trace| \cite{simtrace-wiki} but
additional firmware for smart card emulation exists, provided by
Osmocom. This firmware is called \verb|cardem|
\cite{simtrace-wiki} and is used for communicating with user
equipment, sending APDU:s either constructed by software or
forwarded from a remotely accessed smart card. The hardware is
capable of tracing and emulating all kinds of ISO 7816
\cite{iso7816} smart cards when the T=0 protocol is used
\cite{simtrace-wiki} but the focus is on SIM.

\subsection{Raspberry Pi 4}

Two Raspberry Pi 4 units are used due to the ubiquity and
portability of the hardware platform. The main computer
requirements based on the proposed system are two Linux-based
devices with USB and IP network connectivity capabilities,
rendering most computer systems a viable choice where the choice
of Raspberry Pi is due to the small form factor, inexpensive
hardware and lower power drain, though none of those properties
are a requirement.

\subsection{Teltonika RUT200 v2.0}

The Teltonika RUT200 v2.0 is used as provisional user equipment
the SIMtrace 2 hardware running \verb|cardem| connects to, meaning
the receiver of forwarded APDU:s from a remotely accessed smart
card. The device provides 2G, 3G and 4G capabilities for mobile
network connectivity, authenticated to via the remotely accessed
smart card and has a WiFi and Ethernet interface allowing wired
and wireless connections and internet access, acting as a gateway
between the mobile network and the local network.

\subsection{Smart card readers, SIM, server and router}

Generic physical smart card readers are used for enabling access to
smart cards for the Raspberry Pi which will forward communications
from the smart card over IP to a destination device, e.g. a server
or a peer Raspberry Pi connected to a SIMtrace unit and user
equipment.

A server, virtual or not, is used as a central connectivity hub
enabling rerouting of SIM-to-UE communication to a either a
different UE or a different SIM. A basic router will be used to
provide IP network access for the Raspberry Pi and internet access
to the aforementioned connectivity hub.

\section{Prototype development}
\label{sec:prototype-development}

The initial development for explorative purposes is to create a
prototype \cite[62]{sommerville-software-engineering} \cite[56]{thomas-hunt-pragmatic-programmer} where ideas leading to
requirements, what to test for in a benchmarking and functionality
experiment which drives the decisions behind the final
implementation will be conducted. Requirements elicitation and
experimentation, described below, will use knowledge acquired in
the prototyping phase while also themselves provide additional
information that can sprout new concepts to initially implement in
the prototype before adding to the final implementation.

\section{Requirements elicitation}
\label{sec:requirements-elicitation}

Gather requirements, based on what the company wants to
accomplish and technical requirements based on specifications.

Functional and non-functional requirements, source from 
SEP book.

\section{Experimentation}
\label{sec:experimentation}

\section{Implementation}
\label{sec:implementation}

Test-driven development based on requirements and experience from
previous parts (prototype, experiment, requirements elicitation).

\subsection{Design process}

Documentation and architectural decision records (ADR)
\cite{adr, adr-github}.

\subsection{Testing}

Primarily for an explorative implementation of the requirements,
secondarily for remaining reliable.

\subsection{Benchmarking}

For identifying performance differentiations for code changes.
Tests may be benchmarked to see if it's running fast enough.

Reasons why performance may matter (unverified, find a source):
SIM and UE communication may require low latency, especially over
larger networks, considering they are otherwise usually
physically connected.

Most useful in relation to hot code paths and tight loops.

\subsection{Automated tooling}

Tools for documentation generation will be utilized, both for
automatic generation of changelogs that document available
functionality in an operational perspective but also tools that
generate software library documentation from code comments.
Examples of such tools are Doxygen \cite{doxygen} for software
library documentation generation and git cliff \cite{git-cliff} for
changelog generation.

To enable changelog generation, git cliff requires a consistent
format of git commits \cite{git-cliff} called Conventional Commits
\cite{conventional-commits}. This, using git cliff also enables
automatic version numbering \cite{git-cliff-bump-version} by
following the Semantic Versioning (SemVer) specification
\cite{semver}. If required, additional commit types will be created
and documented(in ADR:s?) with custom git parsers added to git
cliff \cite{git-cliff-tips-and-tricks}.

Automated testing.
gtest and gmock (if C++)

Automated benchmarking.
google microbenchmark (if C++)

\subsection{Delimitations}

The project is an initial implementation at Leissner Data for
verifying if it is viable to properly develop a product of this
kind. The intent is a technological exploration and will
therefore contain limited to no user testing. It will also avoid
testing graphical user interfaces (if necessary).

\section{Delimitations}

Only SIM cards are considered. No other smart cards will be used
at this time.
