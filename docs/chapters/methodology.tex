\chapter{Methodology}

\section{Literature study}

\subsection{Compilation of specifications}

To identify the capabilities and technologies usable for
implementation.

\subsection{Existing solutions and usage}

To provide examples of what has been done, how it may be used and
practical notes possibly missed from the specifications.

\subsection{Delimitations (maybe?)}

\section{Development and implementation}

\subsection{Requirements elicitation}

Gather requirements, based on what the company wants to
accomplish and technical requirements based on specifications.

Functional and non-functional requirements, source from 
SEP book.

\subsection{Design process}

Documentation and architectural decision records (ADR)
\cite{adr, adr-github}.

\subsection{Testing}

Primarily for an explorative implementation of the requirements,
secondarily for remaining reliable.

\subsection{Benchmarking}

For identifying performance differentiations for code changes.
Tests may be benchmarked to see if it's running fast enough.

Reasons why performance may matter (unverified, find a source):
SIM and UE communication may require low latency, especially over
larger networks, considering they are otherwise usually
physically connected.

Most useful in relation to hot code paths and tight loops.

\subsection{Automated tooling}

Automated testing.
gtest and gmock (if C++)

Automated benchmarking.
google microbenchmark (if C++)

\subsection{Delimitations}

The project is an initial implementation at Leissner Data for
verifying if it is viable to properly develop a product of this
kind. The intent is a technological exploration and will
therefore contain limited to no user testing. It will also avoid
testing graphical user interfaces (if necessary).

\section{Delimitations}

Only SIM cards are considered. No other smart cards will be used
at this time.
